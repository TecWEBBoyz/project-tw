\section{Progettazione}
\subsection{Struttura Generale}
La navigazione all'interno del sito è organizzata attorno a un menù principale, posizionato nella parte superiore di ogni pagina. Questo menù presenta link alle sezioni pubbliche del sito ed è arricchito da opzioni contestuali che variano a seconda del ruolo dell'utente (visitatore, utente registrato o amministratore).
La struttura del sito è stata pensata per essere intuitiva, con una profondità di navigazione massima di 3 livelli, per garantire che l'utente possa raggiungere qualsiasi contenuto con pochi click.

\subsubsection{Pagine Pubbliche}
\begin{itemize}
    \item \textbf{Home}: è la pagina di destinazione principale del sito. Nella parte superiore, visibile senza scorrimento (l'area "above the fold"), sono presenti:
    \begin{itemize}
        \item Una \textit{call to action} che invita l'utente a scoprire il mondo di Animalia.
        \item Un pulsante per prenotare una visita, che indirizza alla pagina dei servizi.
        \item Un pulsante per esplorare la galleria degli animali.
    \end{itemize}
    Questa impostazione è stata scelta per guidare immediatamente l'utente verso le funzionalità chiave del sito.
    Scorrendo, la pagina presenta una selezione di animali in primo piano, mostrati attraverso delle "card" informative che invogliano l'utente a visitarne la pagina di dettaglio. La pagina si conclude con un \textit{footer} comune a tutto il sito, contenente contatti, orari di apertura e link ai social media.

    \item \textbf{Chi Siamo}: pagina istituzionale che narra la storia, la missione e i valori etici dello zoo. Il layout alterna paragrafi descrittivi a immagini di contenuto che ne rafforzano il messaggio.

    \item \textbf{Servizi}: vetrina delle esperienze offerte dallo zoo. Per ciascun servizio (Safari, Ingresso libero, Visita guidata) viene fornita una descrizione e un pulsante "Prenota ora".
    \begin{itemize}
        \item Se l'utente non è autenticato, il pulsante lo reindirizza alla pagina di \textbf{Login}, per poi riportarlo alla pagina di prenotazione una volta effettuato l'accesso.
        \item Se l'utente è già autenticato, viene portato direttamente al modulo di prenotazione.
    \end{itemize}
    Le immagini in questa sezione sono puramente decorative e sono state implementate come sfondi CSS per non essere indicizzate dagli screen reader, migliorando l'accessibilità.
    
    \item \textbf{Animali}: pagina che raccoglie l'elenco completo degli animali dello zoo. Si presenta come una griglia di schede interattive. Ogni scheda contiene l'immagine, il nome, la specie e l'habitat dell'animale. Al passaggio del mouse, una breve descrizione appare in sovraimpressione. Cliccando su una scheda, l'utente viene reindirizzato alla pagina di dettaglio del singolo animale.
    
    \item \textbf{Dettaglio Animale}: pagina dedicata a un singolo animale. Il layout è a due colonne: a sinistra, una grande immagine e una descrizione testuale completa; a destra, un "identikit" riassuntivo con informazioni chiave come età, dimensioni, dieta e aspettativa di vita, per una consultazione rapida.
    
    \item \textbf{Login}: modulo di accesso per utenti registrati e amministratori. La logica di autenticazione gestisce anche il reindirizzamento: se un utente non autenticato tenta di accedere a una pagina protetta (es. prenotazione), viene prima portato al login e poi, a successo, alla pagina originariamente richiesta.
\end{itemize}
\subsubsection{Pagine Utenti autenticati}
\begin{itemize}
    \item \textbf{Area Personale}: il pannello di controllo dell'utente, dove può visualizzare una tabella con tutte le sue prenotazioni future. Per ogni prenotazione, sono disponibili pulsanti per modificarla o cancellarla. Se non ci sono prenotazioni, un messaggio guida l'utente a esplorare i servizi.

    \item \textbf{Prenotazione Servizio}: modulo per effettuare una nuova prenotazione. L'utente può selezionare il servizio, la data e il numero di partecipanti (con un limite massimo basato sul servizio scelto). È presente anche un campo opzionale per le note.
    
    \item \textbf{Modifica Prenotazione}: modulo, pre-compilato con i dati esistenti, che permette all'utente di aggiornare una prenotazione già effettuata (es. cambiare la data o il numero di persone).
\end{itemize}

\subsubsection{Pagine Amministratore}
\begin{itemize}
    \item \textbf{Dashboard Amministratore}: pagina principale dell'area di gestione. Presenta due tabelle:
    \begin{itemize}
        \item \textbf{Gestione Prenotazioni}: mostra tutte le prenotazioni di tutti gli utenti, con la possibilità di cancellarle.
        \item \textbf{Gestione Animali}: elenca tutti gli animali presenti nel database, con pulsanti per modificarne i dettagli o eliminarli. È inoltre presente un pulsante per aggiungere un nuovo animale.
    \end{itemize}

    \item \textbf{Aggiungi Animale}: modulo completo che permette all'amministratore di inserire un nuovo animale. I campi includono nome, specie, età, habitat, dimensioni, dieta, aspettativa di vita, una descrizione testuale e il caricamento di un'immagine.
    
    \item \textbf{Modifica Animale}: modulo pre-compilato con i dati di un animale esistente. L'amministratore può aggiornare qualsiasi informazione, inclusa la sostituzione dell'immagine.
\end{itemize}
\subsubsection{Pagine di Errore}
\begin{itemize}
    \item \textbf{Errore 404 (Pagina non trovata)}: pagina mostrata quando viene richiesta una URL non valida. Invece di un semplice messaggio di errore, la pagina adotta un tono amichevole e contestualizzato al tema dello zoo (``L'animale che cercavi è scappato!"), fornendo link per tornare alla Home o alla pagina degli animali, migliorando l'esperienza utente.
    \item \textbf{Errore 500 (Errore Interno del Server)}: pagina visualizzata in caso di un errore critico lato server. Anche in questo caso, la grafica è a tema e il messaggio informa l'utente del problema tecnico, suggerendo di riprovare più tardi.
\end{itemize}
\subsection{Header con Barra di Navigazione}
L'header, presente in tutte le pagine, contiene il logo dello zoo e la barra di navigazione principale. Per migliorare l'usabilità e prevenire loop di navigazione, il link alla pagina corrente viene disabilitato. Ad esempio, quando l'utente si trova sulla pagina "Chi Siamo", il relativo link nel menù non è cliccabile. Lo stesso principio si applica al logo: normalmente reindirizza alla Home, ma è disattivato quando ci si trova già in Home.
Su dispositivi mobili, il menù viene compresso in un'icona ``hamburger" che, al click, rivela le voci di navigazione in un pannello a comparsa.

\subsection{Breadcrumb}
Per aiutare l'utente a orientarsi, ogni pagina interna del sito (ad eccezione della Home) include una \textit{breadcrumb} (o ``briciole di pane") nella parte superiore del contenuto. Questo strumento mostra il percorso di navigazione seguito per arrivare alla pagina corrente (es. Home $>$ Animali $>$ Panda). La pagina attuale nel percorso è mostrata come testo semplice non cliccabile per evitare link circolari.

\subsection{Footer}
Il footer è una componente comune a tutte le pagine e funge da punto di riferimento finale per l'utente. Contiene tre sezioni principali:
\begin{itemize}
    \item \textbf{Contatti}: indirizzo, numero di telefono ed email dello zoo, con link diretti (es. ``tel:", ``mailto:").
    \item \textbf{Orari di apertura}: indicazione degli orari di visita.
    \item \textbf{Social Media}: icone che collegano ai profili social dello zoo.
\end{itemize}
