-----------------------------------------------------------------------------------------\\
------------------------------↓↓↓TO BE DONE↓↓↓------------------------------\\
-----------------------------------------------------------------------------------------
\section{Progettazione}
\subsection{Struttura Generale}
La navigazione all'interno del sito è possibile attrverso l'uso di un menu posto nella parte superiore destra della pagina. Questo menù ha degli elementi comuni a tutti gli utenti e altri elementi specifici per l'utente attualmente autenticato.
\subsubsection{Pagine Pubbliche}
\begin{itemize}
    \item \textbf{Home}: la prima pagina visualizzata accedendo al sito.\\
    Nell'area sicura, visualizzata senza necessità di scroll appena l'utente entra nel sito, sono presenti: 
    \begin{itemize}
        \item una call to action, che invita l'utente a approfondire la sua interazione con il sito; 
        \item un pulsasante che invita l'utente a prenotare un servizio dello zoo, e che conduce alla pagina dedicata ai servizi offerti dallo zoo; 
        \item un pulsante che invita l'utente a esplorare la pagina dedicata agli animali dello zoo.
    \end{itemize}
    Inizialmente, il gruppo non aveva previsto l'implementazione di un'area sicura. Tuttavia, in una fase successiva, si è deciso di arricchire questa sezione con una call to action e due pulsanti. Questi pulsanti guidano l'utente verso due elementi chiave del sito: la pagina dedicata ai servizi offerti e quella dedicata agli animali dello zoo.\\
    La pagina \textbf{Home} prosegue quindi con una sezione contenente delle carte che presentano brevemente alcuni degli animali dello zoo, invitando l'utnete a esplorare le pagine ad essi dedicate.\\
    La \textbf{Home} si conclude quindi con il footer comune a tutte le pagine e contente informazioni circa i contatti dello zoo, i suoi orari di apertura e i link alle pagine social dello stesso.
    \item \textbf{Chi siamo}: pagina raggiungibile dal menu principale del sito.\\
    La pagina presenta un desing semplice che accompagna s dei brevi paragrafi dedicati alla storia, ai dipendenti e ai principi etici di rispetto degli animali, delle immagini di contenuto che rappresentano graficamente i contenuti testuali di tali paragrafi.\\
    \item \textbf{Servizi}: pagina dove sono presentati i servizi messi a disposizione dallo zoo.\\
    Con un design simile a quello della pagina \textbf{Chi siamo}, questa scheda presenta i servizi offerti dallo zoo. Tali servizi includono:
    \begin{itemize}
        \item Safari - un giro nel parco con navetta o con la propria macchina;
        \item Ingresso - un ingresso libero, a piedi, nelle zone recintate del parco;
        \item Visita guidata - un ingresso a piedi nel parco, accompagnato da una guida dedicata al gruppo dell'utente;
    \end{itemize}
    Tutti i servizi sono corredati da un pulsante che conduce l'utente autenticato nella pagina di prenotazione, e l'utente non autenticato nella pagina di login (a seguito del quale l'utnete verrà poi reindirizzato nella pagina di prenotazione).\\
    Le immagini che affiancano ciascun servizio sono in questo caso immagini non di contenuto, e sono pertanto state inserite come immagini di sfondo, non esplorabili mediante screen reader.
    %TODO
    \item \textbf{Animali}: la pagina dedicata alla presentazione di tutti gli animali.\\
    Contiene un elenco con tutti gli animali, le loro immagini, il nome della specie e quello dell'animale stesso. Cliccando una qualunque delle immagini si verrà reindirizzati alla
    pagina \textbf{Accedi}
\end{itemize}
%TODO
\subsubsection{Pagine Utenti autenticati}
\begin{itemize}
\item \textbf{Prenota Sevizio}: pagina dedicata agli utenti autenticati, dove \`e possibile prenotare un servizio in una certa data.
\item \textbf{Profilo}: pagina in cui l'utente può visualizzare e gestire i servizi prenotati
\end{itemize}
%TODO
\subsubsection{Pagine Amministratore}
\begin{itemize}
\item \textbf{Admin}: pagina dedicata dedicata alle opzioni di gestione che ha a disposizione l'admin.
\item \textbf{Gestisci Immagini}: pagina dedicata alla gestione delle fotografie presenti nella Homepage. \`E possibile eliminarle,
 ``nasconderle" o accedere alla pagina di modifica delle informazioni testuali quali titolo, descrizione, data e luogo dello scatto.
\item \textbf{Carica immagine}: pagina per il caricamento delle nuove immagini in Homepage. Subito dopo l'inserimento 
si accede alla pagina di modica delle immagini appena caricate.
\item \textbf{Modifica Immagine}: pagina per modificare i dati di un'immagine caricata nel sito.
\item \textbf{Gestione Prenotazioni}: Pagina per visualizzare, confermare o annullare le prenotazioni ricevute dagli utenti registrati.
\end{itemize}
%TODO
\subsubsection{Pagine di Errore}
\begin{itemize}
    \item \textbf{Errore 404:} L'utente ha richiesto una pagina inesistente.
    \item \textbf{Errore 500:} C'\`e stato un errore lato server nella gestione della richiesta.
\end{itemize}
\subsection{Header con barra di navigazione}
L'Header, che fa parte dei Template \textit{Core}, contiene il logo (la firma del fotografo) oltre alla barra di navigazione. Quest'ultima in modalit\`a mobile \`e accessibile tramite \textit{Hamburger Button}.\\
Inoltre, in ogni pagina, viene disattivato il link alla stessa per evitare la creazione di collegamenti circolari. Lo stesso principio è applicato al logo: normalmente, cliccandoci sopra, si viene reindirizzati alla Home, ma quando ci si trova già in Home, il link viene disattivato.
%TODO
\subsection{Breadcrumb}
Il sito integra una breadcrumb in ogni pagina, progettata per aiutare l'utente a rispondere alle domande  \textit{``Dove Sono?"} e \textit{``Come ci sono arrivato?"}. Questo strumento migliora l'esperienza di navigazione, riducendo il possibile disorientamento dell'utente.\\
Il livello di profondità viene mantenuto, mostrando le pagine che precedono quella attuale. La pagina attuale non è cliccabile, evitando così la creazione di link circolari.
%TODO
\subsection{Footer}
Il sito dispone di un footer con i link ai social del fotografo, le informazioni relative ai diritti della pagina, 
e nel quale vengono visualizzati i messaggi toast di conferma/errore.