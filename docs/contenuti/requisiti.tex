\section{Requisiti}

Di seguito sono indicate le specifiche tecniche del progetto individuabili nella pagina ``Regole per il Progetto di Tecnologie Web'' della pagina Moodle del corso, e implementate dal gruppo:
\begin{itemize}
    \item Uso dello standard HTML5 con pagine che degradano in modo elegante e rispettano la sintassi XML;
    \item Layout realizzato con CSS3 puri;
    \item Utilizzo di Flex all'interno delle pagine per garantire una fruizione elegante e piacevole;
    \item Rispetto della completa separazione fra contenuto, presentazione e comportamento attraverso l'uso di template HTML, fogli di stile CSS e file PHP e JavaScript per il comportamento;
    \item Sito accessibile a tutte le categorie di utenti;
    \item Adozione di una struttura semplice con larghezza non superiore a 5 e profondità non superirore a 3, per garantire facile navigazione;
    \item Presenza di pagine che, mediante script PHP, collezzionano contenuti inserite dall'utente, permettendone cancellazione e modifica;
    \item Presenza di un campo di testo libero fra gli input richiesti all'utente;
    \item Adozione di sistemi di controllo dell'input inserito dall'utente sia lato client che lato software;
    \item Utilizzo di database per il salvataggio e l'accesso ai dati inseriti dall'utente;
\end{itemize}