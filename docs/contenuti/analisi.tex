\section{Analisi}

\subsection{Caratteristiche dell'utenza}
Gli utenti che possono accedere al sito si suddividono in tre categorie principali:
\begin{itemize}
    \item \textbf{Utenti generici}: visitatori non registrati o non autenticati, che possono navigare nel sito e consultare liberamente le informazioni generali sugli animali, sugli orari di apertura e sulle attività offerte dallo zoo.
    
    \item \textbf{Utenti registrati}: dopo aver effettuato l'autenticazione, questi utenti possono accedere a funzionalità aggiuntive, tra cui la prenotazione online di visite tradizionali o dell’esperienza del ``Safari``.
    
    \item \textbf{Amministratore}: figura responsabile della gestione dei contenuti del sito. Dopo l’autenticazione, ha accesso a un pannello di controllo attraverso il quale può inserire, modificare o eliminare le schede degli animali presenti nello zoo. L’amministratore ha anche la possibilità di monitorare e gestire le prenotazioni effettuate dagli utenti.
\end{itemize}

\subsection{Funzionalit\`a}
Tutti gli utenti del sito possono:
\begin{itemize}
    \item Visualizzare una pagina contenente tutti gli animali presenti nello zoo;
    \item Esplorare le schede degli animali, leggendo dettagli come nome, specie, descrizione, habitat e altri dati utili;
    \item Consultare la sezione ``Chi siamo``, contenente la storia e la missione dello zoo;
    \item Consultare la sezione ``Servizi``, che descrive le esperienze offerte dallo zoo;
    \item Consultare alcune delle recensioni lasciate dagli altri utenti del sito;
    \item Stampare le pagine informative, se desiderato.
\end{itemize}
Per gli utenti autenticati è inoltre possibile:
\begin{itemize}
    \item Prenotare (per se stesso e altri) una ingresso semplice allo zoo, una vista guidata all'interno dello stesso, o l'esperienza ``Safari``;
    \item Visualizzare le prenotazioni effettuate;
    \item Modificare le prenotazioni esistenti;
    \item Cancellare le prenotazioni effettuate;
    \item Visualizzare la recensione lasciata se presente;
    \item Lasciare una recensione;
    \item Cancellare la recensione;
    \item Modificare la recensione;
    \item Eseguire il logout.
\end{itemize}
Infine, l’amministratore del sito può:
\begin{itemize}
    \item Visualizzare le prenotazioni degli utenti;
    \item Inserire nuove schede relative agli animali presenti nello zoo;
    \item Eliminare animali dal catalogo esistente;
    \item Modificare le informazioni delle schede animali già inserite (nome, descrizione, immagine, ecc.);
    \item Eseguire il logout.
\end{itemize}

\subsection{Target audience}
Il sito si rivolge a un pubblico ampio, composto principalmente da famiglie, scuole, turisti e appassionati di animali interessati a visitare lo zoo ``Animalia'' o a vivere esperienze come il ``Safari''. La piattaforma è pensata per offrire informazioni sugli animali in modo coinvolgente e accessibile, oltre alla possibilità di prenotare facilmente le attività proposte, rendendola adatta anche ai visitatori più giovani accompagnati dai loro genitori.\\
Partcolare attenzione è stata inoltre dedicata all'aspetto dell'accessibilità, al fine di garantire un buon livello di fruibilità per la più vasta gamma possibile di utenti.\\
Il gruppo ha inoltre dedicato molta attenzione alla realizzazione della versione mobile del sito (sopratutto in pagine che richiedono una forte interazione da parte di utenti amminstratori e non), in considerazione del fatto che ormai sempre più spesso gli utenti interagiscono con i siti web attraverso dispositivi mobili.\\
Esempi di ricerche online che potrebbero condurre al sito includono:
\begin{itemize}
    \item zoo per famiglie con bambini;
    \item prenotazione visita zoo Animalia;
    \item prenotazione safari Animalia.
    \item zoo Padova;
    \item zoo Animalia;
    \item zoo vicino a me;
    \item prenotazione zoo;
\end{itemize}
