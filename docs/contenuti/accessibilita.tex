\section{Accessibilit\`a}
\subsection{Panoramica}
Per offrire un'esperienza utente inclusiva e intuitiva, sono state adottate specifiche soluzioni per migliorare l'usabilità e l’accessibilità del sito:
\begin{itemize}
    \item La navigazione è completamente accessibile tramite tastiera.
    \item Sono stati scelti font leggibili, con contrasto adeguato, per garantire una fruizione confortevole a un’ampia varietà di utenti;
    \item In tutte le pagine del sito sono presenti strumenti di supporto alla navigazione, inclusi breadcrumb e jump button, per agevolare l’esperienza dell’utente;
    \item Le immagini non diversamente descritte sono corredate da testi alternativi, facilitando la comprensione per utenti non vedenti o con disabilità visive. Ogni immagine presente nella galleria viene descritta sia da un alt che ne illustra il contenuto, sia da una descrizione inserita dal fotografo per spiegarne il significato.
    \item La struttura del sito è minimale e intuitiva, con una gerarchia poco profonda;
    \item La palette di colori è stata selezionata con particolare attenzione all’accessibilità, garantendo un contrasto elevato tra gli elementi per migliorare la leggibilità e l’usabilità da parte di utenti con disabilità visive.
\end{itemize}

\subsection{Jump Button}
I Jump Button sono stati realizzati come link che diventano visibili quando ricevono il focus tramite la navigazione con il tasto Tab.\\
Questa soluzione è stata adottata per facilitare l’esperienza di utenti che navigano tramite tastiera e di chi utilizza screen reader.
Esempi di Jump Button si hanno nel menu come aiuti alla navigazione, e a fondo pagina come pulsante per tornare in cima.

\subsection{Palette Colori}
La scelta della palette di colori è stata effettuata con attenzione per garantire la conformità agli standard AA delle WCAG. Per verificare i contrasti, è stato utilizzato lo strumento \href{https://coolors.co/contrast-checker}{Color Contrast Checker di Coolors}, con l’obiettivo di mantenere un rapporto di contrasto minimo di 4.5:1 dei colori con lo sfondo.
Lo stesso rapporto non viene raggiunto tra tutte le combianzioni possibili.\\
La palette è stata sottoposta a un’analisi tramite il simulatore di daltonismo disponibile sullo stesso sito.

Per garantire un contrasto adeguato tra le immagini presenti nella Home e i testi sovrapposti (titolo, città e data dello scatto), è stato applicato un gradiente lineare che sfuma dal basso verso l'alto. Inoltre, è stata effettuata una verifica manuale per assicurarsi che il testo rimanga leggibile anche su immagini prevalentemente bianche (tramite il Color Contrast Checker sopra citato).

\begin{figure}[h]
    \centering
    \includegraphics[width=\textwidth]{immagini/palette.png}
    \caption{\href{https://coolors.co/030303-fefefe-f78e69-23b5d3}{link alla palette su coolors.co}}
    \label{fig:palette}
\end{figure}