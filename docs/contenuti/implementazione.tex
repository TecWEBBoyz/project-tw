\section{Implementazione}
L'implementazione del sito ``Animalia" si è basata su un'architettura moderna e modulare, con una netta separazione tra la logica applicativa (PHP), la struttura (HTML), la presentazione (CSS) e l'interattività (JavaScript). Di seguito vengono descritte le principali tecnologie e le scelte implementative adottate.

\subsection{Ambiente di Sviluppo e Produzione con Docker}
Per garantire coerenza tra l'ambiente di sviluppo locale e quello di produzione, oltre a semplificare il processo di deployment, è stato utilizzato Docker. Sono stati creati due file di configurazione \texttt{docker-compose.yml}:
\begin{itemize}
    \item \textbf{\texttt{docker-compose.yml}}: Definisce l'ambiente di sviluppo locale. Avvia un container con un server web Apache e un server database MySQL. I sorgenti del progetto vengono montati come volume per permettere lo sviluppo live senza dover ricompilare l'immagine a ogni modifica.
    \item \textbf{\texttt{docker-compose-prod.yml}}: Simula l'ambiente di produzione. Questo file è configurato per gestire anche un certificato SSL con Let's Encrypt (attraverso un volume \texttt{certs}) e utilizza uno script di setup (\texttt{setup\_apache\_prod.sh}) per configurare l'ambiente in modo ottimale per la produzione.
\end{itemize}
Entrambe le configurazioni includono un servizio \texttt{phpmyadmin} per facilitare la gestione del database. Questo approccio garantisce che il software si comporti in modo prevedibile e riduce drasticamente i problemi di compatibilità.

\subsection{HTML}
Le componenti strutturali delle pagine del sito sono state scritte in \texttt{HTML5}. La scelta di questo standard garantisce la massima compatibilità con i browser moderni e consente l'utilizzo di elementi semantici che migliorano l'accessibilità e l'indicizzazione (SEO) del sito. L'intera struttura delle pagine viene gestita dinamicamente tramite un sistema di templating in PHP.

\subsection{CSS}
Per lo sviluppo dell'interfaccia è stato utilizzato \texttt{CSS3}, che ha permesso di creare layout moderni, responsivi e accessibili.
\subsubsection{Fogli di Stile}
Sono stati adottati due fogli di stile principali per garantire un'esperienza ottimale su vari dispositivi e modalità di visualizzazione:
\begin{itemize}
    \item \textbf{Stile per gli schermi (\texttt{style.css})}: progettato per adattarsi a schermi di diverse dimensioni attraverso l'uso di \textit{breakpoint} e tecniche di responsive design come Flexbox e Grid Layout.
    \item \textbf{Stile di Stampa (\texttt{print.css})}: definito per ottimizzare la visualizzazione dei contenuti quando la pagina viene stampata. Questo foglio di stile nasconde gli elementi di navigazione e interattivi (come pulsanti e menu), dando priorità al contenuto testuale e alle immagini rilevanti per una versione cartacea pulita e leggibile.
\end{itemize}

\subsection{PHP}
Il linguaggio PHP è stato utilizzato come motore principale del backend, gestendo la logica di business, l'interazione con il database e il rendering dinamico delle pagine.

\subsubsection{Templating con TemplateService}
Per ottimizzare la manutenibilità e promuovere il riutilizzo del codice, è stata implementata un'architettura basata su template e componenti. La classe \texttt{PTW/Services/TemplateService} è il cuore di questo sistema e si occupa di:
\begin{itemize}
    \item \textbf{Caricare il template HTML} corrispondente alla pagina richiesta dalla directory \texttt{/static/htmlTemplates/}.
    \item \textbf{Gestire segnaposto (placeholder)}: Sostituisce dinamicamente dei segnaposto (es. \texttt{[[userName]]}) con dati provenienti dal database o elaborati dalla logica PHP.
    \item \textbf{Gestire blocchi ripetuti}: Permette di generare sezioni di HTML ripetute (come liste di animali o prenotazioni) iterando su un array di dati e applicando un sotto-template.
    \item \textbf{Rendering condizionale}: Mostra o nasconde parti dell'HTML in base allo stato dell'utente (es. mostrando il link di "Login" o di "Logout" a seconda che l'utente sia autenticato o meno).
\end{itemize}
Questo approccio garantisce una netta separazione tra logica e presentazione, semplificando gli aggiornamenti futuri e la gestione del codice.

\subsubsection{Autenticazione e Gestione Sessioni (JWT)}
Il sistema di autenticazione si basa su JSON Web Tokens (JWT), un approccio moderno e stateless. La logica è gestita dalla classe \texttt{PTW/Services/AuthService}.
\begin{itemize}
    \item \textbf{Login}: al momento del login, se le credenziali sono valide, viene generato un token JWT contenente l'ID utente, il nome e il ruolo. La libreria utilizzata è \texttt{firebase/php-jwt}.
    \item \textbf{Memorizzazione sicura}: il token viene memorizzato in un cookie \texttt{HttpOnly}, rendendolo inaccessibile da script JavaScript lato client e mitigando così attacchi di tipo XSS.
    \item \textbf{Validazione delle richieste}: per ogni richiesta a pagine protette, il sistema verifica la validità e la scadenza del token presente nel cookie, garantendo l'accesso solo agli utenti autorizzati e con il ruolo corretto (Utente o Amministratore).
\end{itemize}

\subsubsection{Interazione con il Database (Repository Pattern)}
Per l'interazione con il database MySQL è stato adottato il \textit{Repository Pattern}, che astrae la logica di accesso ai dati e la separa dal resto dell'applicazione.
\begin{itemize}
    \item \textbf{\texttt{DBService}}: una classe a basso livello che gestisce la connessione al database tramite PDO e l'esecuzione di query SQL.
    \item \textbf{\texttt{BaseRepository}}: una classe base che implementa le operazioni CRUD (Create, Read, Update, Delete) generiche, riutilizzabili da tutti i repository specifici.
    \item \textbf{Repository Specifici}: per ogni entità del database (es. \texttt{AnimalRepository}, \texttt{BookingRepository}, \texttt{UserRepository}) è stato creato un repository che estende quello base, fornendo metodi specifici per recuperare i dati (es. \texttt{GetElementByUsername}). Questo rende il codice più pulito, organizzato e facile da testare.
\end{itemize}

\subsection{JavaScript}
Il JavaScript è stato utilizzato per migliorare l'interattività e l'esperienza utente.
\subsubsection{Validazione dei Moduli (Form) lato Client}
Nei moduli di inserimento e modifica (es. creazione animale), è stata implementata una validazione lato client per fornire un feedback immediato all'utente. Lo script:
\begin{itemize}
    \item Impedisce l'invio del modulo se i dati non sono validi.
    \item Mostra messaggi di errore specifici accanto ai campi non corretti.
    \item Aggiorna un riepilogo degli errori all'inizio del form, con link interni ai campi errati.
    \item Sposta il focus sul riepilogo degli errori, rendendo più semplice per gli utenti, specialmente quelli che utilizzano screen reader, individuare e correggere i problemi.
\end{itemize}

\subsubsection{Menu Hamburger}
Per i dispositivi mobili, il menu di navigazione principale è accessibile tramite un pulsante ``hamburger". Uno script gestisce l'apertura e la chiusura del menu, modificando le classi CSS per mostrare o nascondere la lista dei link con un'animazione fluida.

\subsection{SQL}
Il database è stato strutturato per rappresentare le entità principali del sistema: utenti, animali, servizi, prenotazioni e recensioni. Le tabelle principali definite nel file \texttt{init.sql} sono:
\begin{itemize}
    \item \textbf{user} (\underline{id}, username, email, telephone, password, role);
    \item \textbf{animal} (\underline{id}, name, species, age, habitat, dimensions, lifespan, diet, description, image);
    \item \textbf{service} (\underline{id}, name, description, price, duration, max\_people);
    \item \textbf{booking} (\underline{id}, user\_id, service\_id, date, num\_people, notes);
    \item \textbf{review} (\underline{id}, user\_id, rating, comment, created\_at, updated\_at);  \textbf{$\leftarrow$ DA TOGLIERE POTENZIALMENTE}
\end{itemize}

% \begin{figure}[H]
%     \centering
%     % Inserire qui l'immagine del nuovo diagramma ER, se disponibile
%     \includegraphics[width=0.9\textwidth]{immagini/database.png} 
%     \caption{Schema Entità-Relazione del database.}
%     \label{fig:db_schema}
% \end{figure}